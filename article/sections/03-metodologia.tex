
\section{Metodologia Experimental}
\label{sec:metodologia}

\subsection{Descrição do Dataset}
\label{subsec:descricao-do-dataset}
O conjunto de dados utilizado neste estudo é o \textit{Fetal Health Classification}, que contém 2.126
registros derivados de exames cardiotocográficos (CTG). O dataset possui 21 \textit{features} que
representam medidas fisiológicas, como indicadores de batimentos cardíacos fetais (FHR), contrações uterinas,
variabilidade cardíaca e variáveis estatísticas baseadas no histograma de FHR (e.g., \textit{baseline value},
\textit{accelerations}, \textit{fetal\_movement}, \textit{histogram\_width}, \textit{histogram\_mode}, etc.).

A variável alvo (\textit{fetal\_health}) é categórica e multinível, com três classes: 1 (Normal), 2 (Suspeito) e 3
(Patológico).
Inicialmente, a coluna alvo foi convertida de tipo \texttt{float64}
para \texttt{int64}.

\subsection{Análise Exploratória de Dados (EDA)}
\label{subsec:analise-exploratoria-de-dados-(eda)}
A Análise Exploratória de Dados (EDA) e o pré-processamento inicial foram conduzidos com o objetivo de inspecionar a
qualidade dos dados e preparar a base para a modelagem.
As etapas de exploração incluíram:

\subsubsection{Ausência de Valores Ausentes e Duplicados}
O dataset não apresentou valores ausentes, eliminando a necessidade de imputation. No entanto, foram
identificados 13 dados duplicados. Como esses dados derivam de exames CTG fisiológicos, as duplicatas não possuíam
significado clínico e foram removidas para evitar o enviesamento de certos modelos, como KNN, Naive Bayes e Redes
Neurais.

\subsubsection{Distribuições e Outliers}
Histogramas foram utilizados para visualizar a distribuição das variáveis numéricas, revelando comportamentos
estatísticos distintos. Muitas \textit{features}, como \textit{fetal\_movement} e
\textit{severe\_decelerations}, mostraram alta concentração de registros próximos de zero.

\begin{figure}[h]
    \centering
    \includegraphics[width=0.9\columnwidth]{../results/figures/histograms.png}
    \caption{Distribuição de todas as variáveis preditoras no dataset (Histogramas).}
    \label{fig:histogramas}
\end{figure}

A análise de \textit{outliers} foi realizada visualmente por meio de \textit{boxplots} e quantitativamente pela técnica
do Intervalo Interquartil (IQR). Observou-se a presença de outliers em várias
\textit{features}. Decidiu-se manter esses valores extremos, pois eles não são considerados ruído,
mas sim representações de eventos clínicos raros, como movimentos fetais intensos ou alterações abruptas no padrão de
variabilidade cardíaca, sendo essenciais para uma modelagem fiel.

\begin{figure}[h]
    \centering
    \includegraphics[width=0.9\columnwidth]{../results/figures/boxplots.png}
    \caption{Boxplots das features, evidenciando a presença de outliers.}
    \label{fig:boxplots}
\end{figure}

\subsubsection{Distribuição da Variável Alvo e Correlação}
A distribuição da variável alvo (\textit{fetal\_health}) foi examinada por meio de um gráfico de barras, confirmando um
forte desbalanceamento com a predominância da classe 1 (Normal). Esse desbalanceamento é comum
em dados clínicos e será considerado na fase de modelagem.

\begin{figure}[h]
    \centering
    \includegraphics[width=0.5\columnwidth]{../results/figures/target_distribution.png}
    \caption{Distribuição das classes da variável alvo (\textit{fetal\_health}).}
    \label{fig:dist_alvo}
\end{figure}

A matriz de correlação (Figura~\ref{fig:correlacao}) revelou um padrão complexo. Embora a maioria das
correlações entre \textit{features} seja baixa (variando entre $\approx -0.25$ e $+0.25$), grupos de variáveis derivadas
do histograma (\textit{histogram\_mean, histogram\_median} e \textit{histogram\_mode}) mostraram correlação quase
perfeita ($\approx 1$). Esse achado sugere que, embora o restante das variáveis seja amplamente
independente (o que favorece algoritmos como Naive Bayes), há redundância dentro do grupo de variáveis do histograma, o
que pode influenciar modelos sensíveis à multicolinearidade, como a Regressão Logística e
Redes Neurais.

\begin{figure}[h]
    \centering
    \includegraphics[width=0.9\columnwidth]{../results/figures/correlation_matrix.png}
    \caption{Matriz de Correlação entre as features.}
    \label{fig:correlacao}
\end{figure}

\subsection{Pré-processamento}
A etapa final de pré-processamento envolveu a normalização dos dados. Optou-se pelo \textbf{StandardScaler} para
transformar cada \textit{feature} para média 0 e desvio padrão 1.

Essa escolha é adequada para os modelos de ML que serão comparados, como KNN (baseado em distância), Regressão Logística
(otimização mais estável) e Naive Bayes Gaussiano (que assume distribuição normal), além de ser mais robusta em cenários
com a presença de \textit{outliers} quando comparada ao Min-Max Scaling. O \textit{scaler}
é ajustado apenas nos dados de treino, dentro do procedimento de validação cruzada, para evitar \textit{data leakage}, e
posteriormente é salvo para uso futuro na fase de modelagem e avaliação.



