
\section{Introdução}

As complicações durante a gravidez representam riscos significativos tanto para a mulher quanto para o desenvolvimento
fetal, tornando a identificação precoce dessas complicações imperativa para intervenções que salvam
vidas \cite{Salini2024}. Uma prática convencional entre obstetras é a análise manual de testes de cardiotocografia (CTG),
um processo que é frequentemente intensivo em trabalho e sujeito a subjetividade e variabilidade
interobservador \cite{Salini2024}.

A Cardiotocografia (CTG) é um exame não invasivo e de baixo custo que monitora o bem-estar fetal, registrando a
frequência cardíaca fetal (FHR) e as contrações uterinas (UC) \cite{Salini2024}. Tradicionalmente, a interpretação da
CTG baseia-se na análise de características como padrões de FHR, acelerações e desacelerações, mas essa interpretação
manual é complexa e pode levar à perda de sinais sutis de sofrimento fetal \cite{Salini2024}.

Para superar as limitações da análise manual, o desenvolvimento de modelos eficientes de classificação da saúde fetal
baseados em aprendizado de máquina (Machine Learning - ML) é crucial para otimizar os recursos médicos e o
tempo \cite{Salini2024}. Os modelos de ML oferecem o potencial para melhorar significativamente a precisão e a
eficiência da avaliação da saúde fetal, auxiliando os obstetras na tomada de decisões informadas \cite{Salini2024}.

Este trabalho propõe a utilização e comparação de cinco algoritmos de classificação em um conjunto de dados de CTG.
Os algoritmos explorados incluem Árvore de Decisão, Vizinhos Mais Próximos (KNN), Naive Bayes, Regressão Logística e
Redes Neurais MLP \cite{Instruções-Projeto-IA}.

O restante deste artigo está organizado da seguinte forma: a Seção \ref{sec:trabalhos_relacionados} discute trabalhos
correlatos na classificação da saúde fetal usando ML. A Seção \ref{sec:metodologia} detalha a metodologia experimental,
incluindo a descrição do conjunto de dados, a análise exploratória e as etapas de pré-processamento e modelagem. Por fim,
as Seções \ref{sec:resultados} e \ref{sec:conclusao} apresentam os resultados e as considerações finais, respectivamente.
