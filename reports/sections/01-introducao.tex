\section{Introdução}

A classificação supervisionada é um dos pilares fundamentais da Aprendizagem de
Máquina. Seu objetivo é mapear exemplos de entrada em categorias previamente definidas,
com base em padrões aprendidos a partir de dados rotulados. Nos últimos anos, técnicas
de classificação tornaram-se amplamente utilizadas em áreas como saúde, finanças,
visão computacional e processamento de linguagem natural.

Neste trabalho, aplicamos algoritmos clássicos de classificação supervisionada em um
dataset público. O objetivo é avaliar o desempenho dos modelos utilizando validação
cruzada, comparar os resultados obtidos e discutir as vantagens e limitações de cada
abordagem.

O artigo está organizado da seguinte forma: na Seção 2 discutimos trabalhos
relacionados; na Seção 3 apresentamos a metodologia experimental; na Seção 4 descrevemos
o dataset; na Seção 5 apresentamos os resultados; na Seção 6 discutimos os achados; e na
Seção 7 apresentamos as considerações finais.
