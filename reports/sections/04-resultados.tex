\section{Resultados}
EXEMPLO:

Os resultados obtidos pelos modelos são apresentados na Tabela \ref{tab:resultados}.
A métrica utilizada foi a acurácia média na validação cruzada.

\begin{table}[H]
    \centering
    \caption{Acurácia média dos modelos na validação cruzada.}
    \label{tab:resultados}
    \begin{tabular}{l c}
        \toprule
        \textbf{Modelo} & \textbf{Acurácia Média} \\
        \midrule
        KNN & XX\% \\
        Naive Bayes & XX\% \\
        Random Forest & XX\% \\
        \bottomrule
    \end{tabular}
\end{table}

Além disso, gráficos comparativos foram gerados para facilitar a interpretação visual
dos resultados (Figura \ref{fig:grafico}).

\begin{figure}[H]
    \centering
    \includegraphics[width=0.7\linewidth]{images/grafico_resultados.png}
    \caption{Comparação visual do desempenho dos modelos.}
    \label{fig:grafico}
\end{figure}
